\chapter{Ασαφή μετρικά ομοιότητας για δεδομένα αισθητήρων}
\label{chap4}
\section{Προκλήσεις}
Όπως είδαμε στο Κεφάλαιο \ref{chap1}, ένας συνεχώς αυξανόμενος αριθμός αισθητήρων ενσωματώνεται στο περιβάλλον του ανθρώπου, συνδέοντας αυτό με το διαδίκτυο.
Ανάλογα με το Διαδίκτυο της Πληροφορίας, και στο \en{IoT} η αναζήτηση αποτελεί μια κομβική λειτουργία, παρέχοντας στους χρήστες τη δυνατότητα εύρεσης αισθητήρων με συγκεκριμένες ιδιότητες.
Οι υπάρχουσες προσεγγίσεις \cite{Nath2007} στηρίζονται κυρίως στην αναζήτηση με βάση κειμενικά μεταδεδομένα, τα οποία χαρακτηρίζουν κάθε μεμονωμένο αισθητήρα (όπως το είδος του, η τοποθεσία εγκατάστασης, η μονάδα μέτρησης, κ.λπ.).
Η συγκεκριμένη μέθοδος παρουσιάζει προβλήματα στην πρακτική εφαρμογή, καθώς συχνά τα μεταδεδομένα των αισθητήρων είναι ελλιπή ή λανθασμένα και απουσιάζει κοινή ορολογία.
\par
Σε αντίθεση με τις παραδοσιακές μεθόδους, μια εναλλακτική προσέγγιση αναζήτησης στο Διαδίκτυο των Πραγμάτων βασίζεται στην ασαφή λογική \cite{Truong2012}. 
Οι κύριοι λόγοι που οδήγησαν στην υιοθέτηση της ασαφούς λογικής στην αναζήτηση στο \en{IoT} είναι:
\begin{itemize}
    \item Η ασαφής λογική αντιμετωπίζει την θορυβώδη και αβέβαιη φύση των δεδομένων των αισθητήρων, οπότε μπορεί να λύσει πιο αξιόπιστα το πρόβλημα της σύγκρισης της ομοιότητας 2 αισθητήρων.
    \item Οι παραδοσιακές μέθοδοι ανάλυσης και σύγκρισης δεδομένων είναι απαιτητικές σε υπολογιστικούς και επικοινωνιακούς πόρους σε αντίθεση με τις μεθόδους της ασαφούς λογικής. Οι πόροι αυτοί, όπως έχουμε αναφέρει στο Κεφάλαιο \ref{chap1}, είναι περιορισμένοι στους κατανεμημένους κόμβους και αισθητήρες που αποτελούν το \en{IoT}.
\end{itemize}
\section{Σχεδιαστικές απαιτήσεις}
Προκειμένου να αντιμετωπιστούν οι παραπάνω προκλήσεις, το βασικό κριτήριο αξιολόγησης ενός συστήματος αναζήτησης στο \en{IoT} είναι η δυνατότητα του να διαχειριστεί μεγάλο αριθμό αισθητήρων.
Αυτός ο αριθμός είναι συχνά άγνωστος κατά τη φάση σχεδιασμού του συστήματος και μπορεί να κυμαίνεται σε πολλαπλές τάξεις μεγέθους.
Αυτό σημαίνει ότι η απόδοση του συνολικού συστήματος αναζήτησης εξαρτάται άμεσα από την υπολογιστική πολυπλοκότητα της βασικής επιχείρησης, δηλαδή της σύγκρισης δύο αισθητήρων.
Ειδικότερα, το κόστος επικοινωνίας μεταξύ των αισθητήρων και της κεντρικής μονάδας επεξεργασίας πρέπει να ελαχιστοποιείται, καθώς οι αισθητήρες χαρακτηρίζονται από σημαντικούς περιορισμούς στους ενεργειακούς και υπολογιστικούς πόρους.
Ο υπολογισμός και η αποθήκευση μιας σύνοψης των δεδομένων που έχει συλλέξει ο αισθητήρας μειώνει σημαντικά τους πόρους που απαιτούνται για τη μεταφορά τους. Παράλληλα, επιταχύνει τον διαμοιρασμό και τη καταχώρησή τους σε βάσεις δεδομένων.
Οι συμβατικές μέθοδοι σύγκρισης ροών δεδομένων δεν είναι ενδεδειγμένες για χρήση σε κατανεμημένα περιβάλλοντα αισθητήρων, καθώς η ενεργειακή κατανάλωση αυτών των αλγορίθμων υπερβαίνει κατά πολύ τις διαθέσιμες δυνατότητες των αισθητήρων χαμηλής ενεργειακής κατανάλωσης.
Τέλος, η σύγκριση δύο αισθητήρων οφείλει να είναι ανθεκτική (ροβούστ), ώστε να ίναι ικανή να αναγνωρίζει παρόμοια μοτίβα και τάσεις στις χρονοσειρές δεδομένων τους, ακόμη και όταν υπάρχουν διαφορές στις απόλυτες τιμές των μετρήσεων.
\section{Συναφείς εργασίες} \label{Related works}
Στη συνέχεια, θα παρουσιαστεί το σύστημα ασαφούς αναζήτησης\cite{Truong2012}\cite{Truong2013}, το οποίο αποτέλεσε την βάση και έμπνευση αυτής της διπλωματικής εργασίας.
\par
\subsection{Βασικές αρχές ασαφούς αναζήτησης}
Μια βασική λειτουργία της προσέγγισης είναι η κατασκευή ενός ασαφούς συνόλου από μια ροή δεδομένων.
Έστω \(S\) ένας οποιοσδήποτε αισθητήρας, \(U_S\) το σύνολο των μετρήσεων του, \(S(t_i), i=0..|U_S|\) οι επιμέρους μετρήσεις του αισθητήρα \(S\) και \(F_S\) το κατασκευαζόμενο ασαφές σύνολο. Στόχος είναι να καθορισθεί μία συνάρτηση συμμετοχής \( \mu_S(x): U_S \rightarrow [0,1]\) που να αναπαριστά επαρκώς τη χρονοσειρά δεδομένων που παρήγαγε ο αισθητήρας.
\par
Ο στόχος είναι να αποκτήσουμε μια προσέγγιση της κατανομής των τιμών του αισθητήρα.
Αυτή η προσέγγιση θα προκύψει από τον υπολογισμό της πυκνότητας των τιμών του αισθητήρα γύρω από κάθε τιμή.
Έστω \(x\SPSB{\en{S}}{\en{min}}\) και \(x\SPSB{\en{S}}{\en{max}}\) η μικρότερη και μεγαλύτερη τιμή των μετρήσεων του αισθητήρα \(S\), αντίστοιχα.
Δεδομένου ενός διαστήματος \(\Delta{x} = [x-r, x+r] \subset [x\SPSB{\en{S}}{\en{min}}, x\SPSB{\en{S}}{\en{max}}]\) για \(r > 0\), η πυκνότητα του πληθυσμού των μετρήσεων του αισθητήρα είναι ανάλογη με το πόσες μετρήσεις \(x \in U_{\en{S}}\) ανήκουν στο \(\Delta{x}\) σε ένα χρονικό διάστημα \(\Delta{t}\), όταν \(r \rightarrow 0 \) και το διάστημα \(\Delta{x}\) σαρώνει ολόκληρο το πεδίο τιμών \([x\SPSB{\en{S}}{\en{min}}, x\SPSB{\en{S}}{\en{max}}]\).
Συγκεκριμένα, ορίζουμε ως πυκνότητα γειτονιάς του \en{x}:
\begin{equation} \label{eq:4.1}
    ndg^S(x) = \sum_{i=1}^{|U_S|} e^{-\left[ \frac{2d_E(x, S(t_i))}{r} \right]^2}
\end{equation}
, όπου \(d_E\) η Ευκλείδεια απόσταση μεταξύ 2 τιμών.
Εντέλει, η συνάρτηση συμμετοχής \(\mu_S(x)\) ισούται με την \({ndg^S(x)}\) κανονικοποιημένη στο διάστημα (0,1) και το τελικό ασαφές σύνολο είναι 
\( F_S = \{(x, \mu_S(x)) | x \in U_S\}\). 
\par
Στη συνέχεια, με βάση τις κατασκευασμένες συναρτήσεις συμμετοχής, θέλουμε να υπολογίσουμε ένα μετρικό ομοιότητας των αισθητήρων βασισμένο στις τιμές των δεδομένων που παράγουν. 
Έστω, λοιπόν, 2 αισθητήρες τοποθετημένοι σε 2 διαφορετικές τοποθεσίες, \(Α\) και \(Β\). 
Για κάθε αισθητήρα, έχουμε υπολογίσει το ασαφές σύνολο από τις μετρήσεις του, δηλαδή έχουμε τα \(F_{A}=\{(x,\mu_{A}(x))|x\in\mathbb{R}\}\) και \(F_{Β}=\{(x,\mu_{Β}(x))|x\in\mathbb{R}\}\).
Έστω ότι έχουμε έναν τρίτο αισθητήρα \(S\) και θέλουμε να υπολογίσουμε ένα μετρικό ομοιότητας μεταξύ του \(S\) και των \(Α\) και \(Β\).
Εάν πάρουμε δειγματοληπτικά μια μέτρηση \(x\in{U_S}\), οι συναρτήσεις συμμετοχής των \(F_A\) και \(F_B\) θα μας δώσουν τον βαθμό συμμετοχής του \(x\) στα 2 ασαφή σύνολα.
Με βάση τα παραπάνω ορίζουμε ως μετρικό ομοιότητας του αισθητήρα \(S\) σε σχέση με τον αισθητήρα \(V\) ως εξής:
\begin{equation} \label{eq:4.2}
    \Phi_{S}(V) = \frac{1}{\delta(S,V)}\frac{1}{|U_S|}\sum_{x\in{U_S}}\mu_{V}(x)
\end{equation}
όπου \(\delta(S, V)\) ονομάζουμε την διαφορά των ευρών των αισθητήρων \(S\) και \(V\).
Υπολογίζεται ως εξής:
\begin{equation}
    \delta(S, V) = |q\SPSB{\en{S}}{1} - q\SPSB{\en{V}}{1}| + |q\SPSB{\en{S}}{3} - q\SPSB{\en{V}}{3}|
\end{equation}
όπου \( q\SPSB{\en{S}}{1}, q\SPSB{\en{S}}{3} \in U_S \) και  \( q\SPSB{\en{V}}{1}, q\SPSB{\en{V}}{3} \in U_V \) είναι τα πρώτα και τρίτα τεταρτημόρια της κατανομής των τιμών των αισθητήρων \(S\) και \(V\).
Τα τεταρτημόρια ενός συνόλου ταξινομημένων τιμών είναι τα 3 σημεία που χωρίζουν το σύνολο σε 4 ίσα σύνολα, καθένα εκ των οποίων αντιπροσωπεύει το 25\% του πληθυσμού των τιμών.
Η χρήση της διαφοράς των ευρών των αισθητήρων γίνεται για 2 λόγους: (1) για να αποκλείσει αισθητήρες διαφορετικού τύπου ή αισθητήρες ενσωματωμένους σε εντελώς διαφορετικό περιβάλλον ή αντικείμενο; και (2) για να ενισχύσει την ομοιότητα μεταξύ αισθητήρων οι οποίοι παράγουν μετρήσεις σε παρόμοια εύρη.
\par
Ωστόσο, η παραπάνω τεχνική παρουσιάζει μια σημαντική έλλειψη: αμελεί πλήρως την χρονική σχέση μεταξύ των μετρήσεων και κάνει την παραδοχή πως δεν έχει σημασία η αλληλουχία των τιμών στην περιγραφή της κατάστασης του περιβάλλοντος ή του αντικειμένου.
Αυτό, προφανώς, δεν ισχύει, καθώς απότομες και ήπιες αλλαγές των τιμών του αισθητήρα σηματοδοτούν διαφορετικές καταστάσεις.
Η χρονική μεταβολή των μετρήσεων αποτυπώνεται από την διακριτή χρονική παράγωγο των τιμών, η οποία ορίζεται στο χρονικό σημείο \(t_i\) ως:
\begin{equation}
    S'(t_i) = \frac{S(t_{i+1})-S(t_i)}{t_{i+1}-t_i}
\end{equation}

Στη συνέχεια, ορίζουμε το σύνολο των διακριτών παραγώγων του \(S\) ως \(U_{S'} = \{x'= S'(t_i)|i=1..|U_S|-1\}\).
Χρησιμοποιώντας τα παραπάνω και την εξίσωση \ref{eq:4.1}, το ασαφές σύνολο των διακριτών παραγώγων του \(S\) ορίζεται ως \(F_{S'}=\{(x',\mu_{S'}(x'))|x'\in U_{S'}\}\).
Λαμβάνοντας υπόψη και τις χρονικές παραγώγους, το μετρικό ομοιότητας μεταξύ 2 αισθητήρων \(S\) και \(V\) πλέον ορίζεται ως εξής:
\begin{equation}
    \Phi_S(V) = \frac{1}{\delta(S,V)}\frac{1}{|U_S|}\sum^{|U_S|}_{i=1}\mu_V(S(t_i)) \times \mu_{V'}(S'(t_i))
\end{equation}
\subsection{Μετρικά ομοιότητας ασαφών συνόλων}

\section{Μεθοδολογία}
Όπως δείξαμε παραπάνω, η χρήση της ασαφούς λογικής λύνει πολλά από τα ζητούμενα της αναζήτησης στο \en{IoT}.
Εντούτοις, ο συνδυασμός αυτός δεν έχει μελετηθεί εκτενώς και υπάρχουν σημαντικά περιθώρια βελτίωσης των εφαρμογών.
Η συνεισφορά αυτής της διπλωματικής εργασίας αφορά επεκτάσεις στο σύστημα που περιγράφηκε στο υποκεφάλαιο \ref{Related works}, και συγκεκριμένα στην κατασκευή του ασαφούς συνόλου και στα μετρικά ομοιότητας.

\subsection{Δημιουργία Ασαφούς Συνόλου}

Η διαδικασία κατασκευής ενός ασαφούς συνόλου, όπως περιγράφεται από την εξίσωση \ref{eq:4.1}, είναι μια προσέγγιση της πυκνότητας ενός σήματος.
Παρόλα τα πλεονεκτήματα της προσέγγισης αυτής, το υπολογιστικό κόστος είναι μεγάλο, ειδικά όταν \(r \rightarrow 0\) και το εύρος του σήματος είναι μεγάλο.
Για αυτόν τον λόγο, ως προσέγγιση της πυκνότητας ενός σήματος χρησιμοποιήθηκε το ιστόγραμμα των τιμών του, είτε στην απόλυτη μορφή του ως καταμέτρηση των τιμών, είτε κανονικοποιημένο ως πυκνότητα πιθανότητας. 

\subsection{Μετρικά Ομοιότητας}
Με βάση το θεωρητικό υπόβαθρο του Κεφαλαίου \ref{chap3} και την παραπάνω παρουσίαση της χρήσης της ασαφούς λογικής στο πρόβλημα της αναζήτησης αισθητήρων στο \en{IoT}, η σύγκριση 2 ασαφών συνόλων που αποτυπώνουν ροές δεδομένων μπορεί να γίνει με πολλαπλούς τρόπους.
Συγκεκριμένα, η βιβλιογραφία καταγράφει πληθώρα μετρικών ομοιότητας 2 ασαφών συνόλων, οι οποίες έχουν χρησιμοποιηθεί επιτυχώς σε παρόμοιες εφαρμογές.
\par
Οι κυριότερες μετρικές που εξετάζονται στην παρούσα εργασία κατηγοριοποιούνται ως εξής:

\subsubsection{Συνολοθεωρητικές Μετρικές}

Οι συνολοθεωρητικές μετρικές βασίζονται στις κλασικές πράξεις συνόλων επεκταμένες για ασαφή σύνολα. Για δύο ασαφή σύνολα \(A\) και \(B\) με συναρτήσεις συμμετοχής \(\mu_A(x)\) και \(\mu_B(x)\):

\begin{itemize}
    \item \textbf{Δείκτης \en{Jaccard} (Συντελεστής \en{Tanimoto})}:
    \[J(A,B) = \frac{\sum_{i} \min(\mu_A(x_i), \mu_B(x_i))}{\sum_{i} \max(\mu_A(x_i), \mu_B(x_i))}\]
    Μετρά τον λόγο της ασαφούς τομής προς την ασαφή ένωση των συνόλων.
    
    \item \textbf{Συντελεστής \en{Dice} (\en{Sørensen-Dice})}:
    \[D(A,B) = \frac{2\sum_{i} \min(\mu_A(x_i), \mu_B(x_i))}{\sum_{i} \mu_A(x_i) + \sum_{i} \mu_B(x_i)}\]
    Δίνει διπλή βαρύτητα στην τομή σε σχέση με το άθροισμα των πληθικοτήτων.
    
    \item \textbf{Συντελεστής Επικάλυψης (\en{Szymkiewicz-Simpson})}:
    \[O(A,B) = \frac{\sum_{i} \min(\mu_A(x_i), \mu_B(x_i))}{\min(\sum_{i} \mu_A(x_i), \sum_{i} \mu_B(x_i))}\]
    Κανονικοποιεί την τομή με την πληθικότητα του μικρότερου συνόλου.
\end{itemize}

\subsubsection{Μετρικές Απόστασης}

Οι μετρικές απόστασης υπολογίζουν την διαφορά μεταξύ των συναρτήσεων συμμετοχής και μετατρέπονται σε ομοιότητες:

\begin{itemize}
    \item \textbf{Ευκλείδεια Απόσταση}:
    \[d_{Eucl}(A,B) = \sqrt{\sum_{i} (\mu_A(x_i) - \mu_B(x_i))^2}\]
    \[S_{Eucl}(A,B) = \frac{1}{1 + d_{Eucl}(A,B)}\]
    
    \item \textbf{Απόσταση \en{Hamming}}:
    \[d_{Ham}(A,B) = \sum_{i} |\mu_A(x_i) - \mu_B(x_i)|\]
    \[S_{Ham}(A,B) = 1 - \frac{d_{Ham}(A,B)}{n}\]
    όπου \(n\) ο αριθμός των στοιχείων.
    
    \item \textbf{Απόσταση \en{Chebyshev}}:
    \[d_{Cheb}(A,B) = \max_{i} |\mu_A(x_i) - \mu_B(x_i)|\]
    \[S_{Cheb}(A,B) = 1 - d_{Cheb}(A,B)\]
    Χρησιμοποιεί τη μέγιστη διαφορά σε οποιοδήποτε σημείο.
\end{itemize}

\subsubsection{Μετρικές Συσχέτισης}

Οι μετρικές συσχέτισης αξιολογούν την γραμμική και γεωμετρική σχέση των συναρτήσεων συμμετοχής:

\begin{itemize}
    \item \textbf{Ομοιότητα \en{Cosine}}:
    \[C(A,B) = \frac{\sum_{i} \mu_A(x_i) \cdot \mu_B(x_i)}{\|\mu_A\| \cdot \|\mu_B\|}\]
    όπου \(\|\mu_A\| = \sqrt{\sum_{i} \mu_A(x_i)^2}\) η ευκλείδεια νόρμα.
    
    \item \textbf{Συντελεστής \en{Pearson}}:
    \[\rho(A,B) = \frac{\sum_{i} (\mu_A(x_i) - \overline{\mu_A})(\mu_B(x_i) - \overline{\mu_B})}{\sqrt{\sum_{i} (\mu_A(x_i) - \overline{\mu_A})^2} \sqrt{\sum_{i} (\mu_B(x_i) - \overline{\mu_B})^2}}\]
    όπου \(\overline{\mu_A}\) και \(\overline{\mu_B}\) οι μέσες τιμές των συναρτήσεων συμμετοχής.
    
    \item \textbf{Διασταυρούμενη Συσχέτιση (\en{Cross-Correlation})}:
    \[CC(A,B) = \max_{\tau} \sum_{i} \mu_A(x_i) \cdot \mu_B(x_{i+\tau})\]
    Βρίσκει τη βέλτιστη χρονική μετατόπιση για μέγιστη συσχέτιση.
\end{itemize}

\subsubsection{Πληροφοριοθεωρητικές Μετρικές}

Οι πληροφοριοθεωρητικές μετρικές αντιμετωπίζουν τις συναρτήσεις συμμετοχής ως κανονικοποιημένες κατανομές πιθανοτήτων:

\begin{itemize}
    \item \textbf{Απόκλιση \en{Jensen-Shannon}}:
    \[JS(P||Q) = \frac{1}{2}D_{KL}(P||M) + \frac{1}{2}D_{KL}(Q||M)\]
    όπου \(M = \frac{P+Q}{2}\) και \(D_{KL}(P||Q) = \sum_{i} P(x_i) \log \frac{P(x_i)}{Q(x_i)}\).
    \[S_{JS}(A,B) = 1 - \sqrt{JS(P_A||P_B)}\]
    
    \item \textbf{Συντελεστής \en{Bhattacharyya}}:
    \[BC(P,Q) = \sum_{i} \sqrt{P(x_i) \cdot Q(x_i)}\]
    Μετρά την επικάλυψη δύο κανονικοποιημένων κατανομών.
    
    \item \textbf{Απόσταση \en{Hellinger}}:
    \[d_{Hell}(P,Q) = \frac{1}{\sqrt{2}} \sqrt{\sum_{i} (\sqrt{P(x_i)} - \sqrt{Q(x_i)})^2}\]
    \[S_{Hell}(A,B) = 1 - d_{Hell}(P_A,P_B)\]
\end{itemize}

\subsubsection{Προηγμένες Μετρικές}

Πιο σύνθετες μετρικές που λαμβάνουν υπόψη την δομική και στατιστική φύση των κατανομών:

\begin{itemize}
    \item \textbf{Απόσταση Μεταφοράς Μάζας (\en{Earth Mover's Distance})}:
    Προσεγγίζεται ως απόσταση \(L1\) μεταξύ αθροιστικών κατανομών:
    \[EMD(A,B) \approx \sum_{i} |CDF_A(x_i) - CDF_B(x_i)|\]
    \[S_{EMD}(A,B) = \frac{1}{1 + EMD(A,B)}\]
    
    \item \textbf{Ενεργειακή Απόσταση (\en{Energy Distance})}:
    Βασίζεται στη στατιστική ενέργεια μεταξύ κατανομών:
    \[E(A,B) = 2E[\|X-Y\|] - E[\|X-X'\|] - E[\|Y-Y'\|]\]
    όπου \(X,X' \sim A\) και \(Y,Y' \sim B\) είναι ανεξάρτητα.
    
    \item \textbf{Αρμονικός Μέσος Όρος}:
    \[H(A,B) = \frac{2}{\frac{1}{\sum_i \mu_A(x_i)} + \frac{1}{\sum_i \mu_B(x_i)}} \cdot \frac{\sum_i \min(\mu_A(x_i), \mu_B(x_i))}{\sum_i \max(\mu_A(x_i), \mu_B(x_i))}\]
    Συνδυάζει αρμονικό μέσο των πληθικοτήτων με δείκτη Jaccard.
\end{itemize}

\section{Σύνοψη και Συμπεράσματα}

Στο παρόν κεφάλαιο παρουσιάστηκε η εφαρμογή της ασαφούς λογικής στην αντιμετώπιση του προβλήματος της αναζήτησης αισθητήρων στο Διαδίκτυο των Πραγμάτων. 
Η προσέγγιση αυτή αντιμετωπίζει αποτελεσματικά τις προκλήσεις που θέτει η θορυβώδης και αβέβαιη φύση των δεδομένων αισθητήρων, καθώς και οι περιορισμένοι υπολογιστικοί πόροι των κατανεμημένων συστημάτων \en{IoT}.

Η βασική συνεισφορά του κεφαλαίου είναι η παρουσίαση του αλγορίθμου \en{Normalized Density Gaussian (NDG)} για την κατασκευή ασαφών συναρτήσεων συμμετοχής από ροές δεδομένων, ο οποίος προσφέρει υπολογιστική αποδοτικότητα διατηρώντας παράλληλα την ακρίβεια της αναπαράστασης.
Επιπλέον, αναλύθηκαν διάφορες μετρικές ομοιότητας ασαφών συνόλων, από τις κλασικές συνολοθεωρητικές μέχρι τις πιο σύνθετες πληροφοριοθεωρητικές, παρέχοντας ένα ολοκληρωμένο πλαίσιο για τη σύγκριση αισθητήρων.

Η προσέγγιση αυτή αποτέλεσε την έμπνευση και τη θεωρητική βάση για την ανάπτυξη πιο εξειδικευμένων τεχνικών που εφαρμόζονται στην αναγνώριση ανθρώπινων δραστηριοτήτων από δεδομένα υγείας, όπως θα αναλυθεί λεπτομερώς στο επόμενο κεφάλαιο.
Η μετάβαση από τη γενική αναζήτηση αισθητήρων στο \en{IoT} στην εξειδικευμένη ανάλυση δεδομένων υγείας αποδεικνύει τη ευελιξία και την ευρύτερη εφαρμοσιμότητα των ασαφών μεθόδων στην επεξεργασία δεδομένων αισθητήρων. 

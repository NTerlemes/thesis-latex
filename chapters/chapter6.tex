\chapter{Αποτελέσματα}

Σε αυτό το κεφάλαιο, θα παρουσιασθούν και θα αναλυθούν τα αποτελέσματα των πειραμάτων, όπως πρόκυψαν από τα βασικά ερευνητικά ερωτήματα και παρουσιάστηκαν στο Κεφάλαιο 5.

\section{Ερευνητικό Ερώτημα 1}

# Να μπει σε italic και ελληνικά:
# #### **RQ1: Can we make membership function computation efficient enough for real-time use?**
# **The Question:** Traditional KDE is too slow for IoT applications. Can our NDG-S algorithm achieve significant speedup while maintaining accuracy?

\subsection{Αναγνώριση βέλτιστων παραμέτρων για την πειραματική μέθοδο \en{NDG-S}}

\subsection{Συγκρίση αποδοτικότητας πειραματικής μεθόδου \en{NDG-S} με την \en{state-of-art} μέθοδο \en{KDE}}

\section{Ερευνητικό Ερώτημα 2}

# Να μπει σε italic και ελληνικά:
# ### **RQ2: Can per-sensor fuzzy membership functions improve activity recognition?**
# **The Question:** Traditional approaches treat all sensors equally. Can preserving individual sensor characteristics improve performance?

/subsection{Πειράμα αναγνωρίσης δραστηριότητας στο \en{Opportunity dataset} }

/subsection{Πειράμα πειραμάτος αναγνωρίσης δραστηριότητας στο \en{PAMAP2 dataset} }

\section{Ερευνητικό ερώτημσ 3}

#### **RQ3: Do our similarity metrics work consistently across different datasets?**
**The Question:** Will metrics that work well on one dataset also work on datasets with different sensors and sampling rates?

/subsection{Πειράμα αναγνωρίσης δραστηριότητας μεταξύ των \en{Opportunity} και \en{PAMAP2 datasets} }